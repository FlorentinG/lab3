\documentclass[11pt,a4paper]{report}

\usepackage[T1]{fontenc}
\usepackage[utf8]{inputenc}
\usepackage[english]{babel}
\usepackage{lmodern}
%\usepackage{circuitikz}
\usepackage{color}
\usepackage{wrapfig}
\usepackage{placeins}
\usepackage{subfigure}
\usepackage{tabu}
\usepackage{fullpage}
\usepackage[squaren]{SIunits}
\usepackage{graphicx}
%\usepackage[pdftex]{graphicx}
\usepackage{epstopdf}
\usepackage{epsfig}
\usepackage{hyperref}
\usepackage{tikz}
\usepackage{tikz-qtree}
\usepackage{eurosym}
%\usepackage{chemist}
\usepackage{amsmath}
\usepackage{amssymb}
\usepackage{mathrsfs}
\usepackage{dsfont}% use $\mathds{1}$
\newcommand{\C}{\mathbb{C}}
\newcommand{\N}{\mathbb{N}}
\newcommand{\Z}{\mathbb{Z}}
\newcommand{\R}{\mathbb{R}}
\newcommand{\red}{\textcolor{red}}
\newcommand{\dis}{\displaystyle}
\newcommand{\dr}{\partial}
\newcommand{\txt}{\text}
\newcommand{\td}{\todo[inline]}
\newcommand{\ttt}{\texttt}
\newcommand{\itt}{\textit}

\usepackage{algorithm}
\usepackage{todonotes}
\usepackage[noend]{algpseudocode}

%\newtheorem{theoreme}			     {Théorème}	[chapter]
%\newtheorem{proposition}[theoreme]	 {Proposition}	
%\newtheorem{corollaire}	  [theoreme]	 {Corollaire}	
%\newtheorem{lemme}	      [theoreme]  {Lemme}		
%\newtheorem{definition}	         {Définition}[chapter]
%\theoremstyle{definition}
%\newtheorem{exemple}			     {Exemple}	[chapter]
%\newtheorem{contreexemple}[exemple]{Contre-exemple}
%\newtheorem{probleme}	             {Probl\`eme}[chapter]

\usepackage{listings}
\usepackage{textcomp}
\definecolor{listinggray}{gray}{0.9}
\definecolor{lbcolor}{rgb}{0.9,0.9,0.9}
\lstset{
	backgroundcolor=\color{lbcolor},
	tabsize=4,
	rulecolor=,
	language=matlab,
        basicstyle=\scriptsize,
        upquote=true,
        aboveskip={1.5\baselineskip},
        columns=fixed,
        showstringspaces=false,
        extendedchars=true,
        breaklines=true,
        prebreak = \raisebox{0ex}[0ex][0ex]{\ensuremath{\hookleftarrow}},
        frame=single,
        showtabs=false,
        showspaces=false,
        showstringspaces=false,
        identifierstyle=\ttfamily,
        keywordstyle=\color[rgb]{0,0,1},
        commentstyle=\color[rgb]{0.133,0.545,0.133},
        stringstyle=\color[rgb]{0.627,0.126,0.941},
}

\DeclareMathOperator{\e}{e}

\title{Titre}
\author{Florentin Goyens}
\date{\today}

\begin{document}
\tabulinesep=1.2mm
\begin{center}
\hrule
\begin{tabular}{c}
\\[0.005cm]
\Large{Applied Numerical Methods - Lab 3}\\[0.3cm]
\textsc{Goyens} Florentin  \& \textsc{Weicker} David\\[0.2cm]
$\text{13}^{\text{th}}$ October 2015\\[0.2cm]
\end{tabular}
\hrule
\end{center}


\section{Stationary heat conduction in 1-D}

In a one dimensional pipe we are interested in the temperature evolution along the z-axis.  We will study the behaviour of the numerical solution based on finite differences.



\subsection*{Distretization to a linear system of equations}

The z-axis is discretization in $N+1$ points spreading from $0$ to $L$. the differential equation for the temperature is the following
$$-\kappa \dfrac{d^{2}T}{dz^{2}} +v\rho C\dfrac{dT}{dz}=Q(z)$$
with $$Q(z)= \left\{ \begin{array}{lll}
0 & if & 0\leq z <a\\
Q_{0}\sin \Big(\dfrac{z-a}{b-a}\pi \Big) & if & a \leq z \leq b\\
0 & if & b\leq z \leq L.\\
\end{array}\right.$$
We assumed that $\kappa$ as a constant value through the pipe.

The boundary conditions are
$$T(0)=T_{0}$$ and $$-\kappa \dfrac{dT}{dz}(L)=k_{v}(T(L)-T_{out}).$$

For the discretization, let $T_{0}, T_{1}, \dots, T_{N}$ be the unknown temperature we will solve for. The step is $h=L/N$. Define $z_{i}=i*h$ and $T_{i}\approx T(z_{i})$. It is clear that $z_0 =0$ and $x_N =L$. We also define the ghost point $x_{N+1}=L+h$ that will be used implicitly to write the system. The real unknowns we will include in the system are  $T_{1}, \dots, T_{N}$. $T_{0}$ is known from the start. 


With finite difference approximation we rewrite the problem as 
$$-\kappa \dfrac{T_{i+1}-2T_{i}+T_{i-1}}{h^{2}}
+ v\rho C\dfrac{T_{i+1}-T_{i-1}}{2h}=Q(z_{i}),$$
for all the points, $i=1, \dots N$. This is equivalent to
$$\underbrace{\Big(-\dfrac{v\rho Ch}{2}  -\kappa    }_{\alpha}\Big) T_{i-1} 
+ \underbrace{\Big(2k\Big)}_{\beta} T_{i-1}
+ \underbrace{\Big( \dfrac{v\rho Ch}{2}  -\kappa    }_{\gamma}\Big) T_{i+1}
=h^{2}Q(z_{i}),$$


As for the boundary condition we have 
the straightforward $$T_{0}=400,$$
as well as
$$-\kappa \dfrac{T_{N+1}-T_{N-1}}{h}=k_{v}(T_{N}-T_{out}).$$
This allows to express $T_{N+1}$ with other variables and remove it from the system of equations.

$$T_{N+1}=T_{N-1}-\dfrac{2hk_{v}}{k}T_{N}+\dfrac{2hk_{v}}{k}T_{out}.$$


This is a linear system of $N+1$ equations. As we would like to avoid dividing by the small value $h^2$, we rewrite the system as follows. This system is tridiagonal and the Matlab \textit{band solver} will be very (very) efficient is the matrix is defined as sparse.

\end{document}